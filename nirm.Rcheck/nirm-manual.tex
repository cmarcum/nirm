\nonstopmode{}
\documentclass[letterpaper]{book}
\usepackage[times,inconsolata,hyper]{Rd}
\usepackage{makeidx}
\usepackage[utf8,latin1]{inputenc}
% \usepackage{graphicx} % @USE GRAPHICX@
\makeindex{}
\begin{document}
\chapter*{}
\begin{center}
{\textbf{\huge Package `nirm'}}
\par\bigskip{\large \today}
\end{center}
\begin{description}
\raggedright{}
\item[Type]\AsIs{Package}
\item[Title]\AsIs{Network Inductive Reasoning Model}
\item[Version]\AsIs{0.5.5}
\item[Date]\AsIs{2012-01-12}
\item[Author]\AsIs{Raffaele Vardavas and Christopher Steven Marcum}
\item[Maintainer]\AsIs{Christopher Steven Marcum }\email{cmarcum@uci.edu}\AsIs{}
\item[Description]\AsIs{Fits Vardavas's inductive reasoning model to
theoretical or empirical networks.}
\item[License]\AsIs{GPL (>= 2.0)}
\item[Depends]\AsIs{R (>= 2.12), igraph, deSolve}
\item[LazyLoad]\AsIs{yes}
\end{description}
\Rdcontents{\R{} topics documented:}
\inputencoding{utf8}
\HeaderA{nirm-package}{Network Inductive Reasoning Model}{nirm.Rdash.package}
\keyword{package}{nirm-package}
%
\begin{Description}\relax
Fits Vardavas's Inductive Reasoning Model to a Network.
\end{Description}
%
\begin{Details}\relax

\Tabular{ll}{
Package: & informR\\{}
Type: & Package\\{}
Version: & 1.0\\{}
Date: & 2011-04-17\\{}
License: & GPL 2.0 or greater\\{}
LazyLoad: & yes\\{}
}
Use this package to fit a an inductive reasoning model of a binary decision process to a network. 
\end{Details}
%
\begin{Author}\relax
Author: Raffaele Vardavas and Christopher Steven Marcum <cmarcum@uci.edu>
\end{Author}
%
\begin{SeeAlso}\relax
\LinkA{nirm}{nirm}
\end{SeeAlso}
\inputencoding{utf8}
\HeaderA{nirm}{Network Inductive Reasoning Model.}{nirm}
%
\begin{Description}\relax
Generate numeric categories of events in an idXevent list.
\end{Description}
%
\begin{Usage}
\begin{verbatim}
nirm(network.model=c("Erdos","Watts","NW.Watts","Barabasi","Empirical"),
               N=10^3,M=20,behavioral.model="Simple_Personal",enet=NULL,usbm=NULL,
               hyper.params=list(model=c(R0=2,gamma=1/3,iip=.02,eff=1,tfs=300,s.mem=.7),
                          vattr=cbind(V=rep(0,N-1),w=rep(0.3,N-1),inf.nn=rep(0,N-1),vacc.nn=rep(0,N-1))),
               added.RMCN=FALSE, RMCN.R0=0,output.dir="",save.model=TRUE,verbose=FALSE,seed=NULL)
\end{verbatim}
\end{Usage}
%
\begin{Arguments}
\begin{ldescription}
\item[\code{network.model}] A character string indicating the network generation process. Default is "Erdos." See Details.
\item[\code{N}] An integer for the size of the network draw. If enet is non-null, N is taken from enet. Default is 10\textasciicircum{}3.
\item[\code{M}] An integer for the average number of edges per vertice. Default is 20.
\item[\code{behavioral.model}] A character string indicating the behavioral component of the model. Default is "Simple\_Personal." See Details.
\item[\code{enet}] If network.model="Empirical", an igraph object representing undirected relational network.
\item[\code{usbm}] Optionally, a list containing the User Specified Behavioral Model. See details.
\item[\code{hyper.params}] A list of parameters for the behavioral and network components of the model. See Details.
\item[\code{added.RMCN}] Logical. Should we add a random-mixing component to the network model? Defaults to FALSE. See Details.
\item[\code{RMCN.R0}] Numeric. If added.RMCN is TRUE, set the R0 for that network. Defaults to 0 (which is not used). Optional.
\item[\code{output.dir}] A character string indicating where to save temporary and (optionally) permanent files. When non-null, this should contain a trailing "/". Default is pwd.
\item[\code{save.model}] Logical. Should we save the final output to an R binary in the output.dir? Defaults to TRUE.
\item[\code{seed}] Optionally, a random seed for the stochastic component of the model
\end{ldescription}
\end{Arguments}
%
\begin{Details}\relax
The \code{network.model} can be any of "Erdos","Watts","NW.Watts","Barabasi", or "Empirical." Each option corresponds to a specific network generation model. 
[[MORE DETAILS ABOUT EACH NETWORK MODEL HERE]]. The user may supply his or her own network data by setting this option to "Empirical" and supplying the appropriate
\pkg{igraph} object in \code{enet}.

The \code{behavioral.model} option specifies how actors in the network weight personal characteristics, information about the world, and the influence of their peers in their behaviors. [[EXPLAIN BEHAVIORAL MODEL HERE]].
Users can supply their own behavioral model by supplying a list of the proper structure to \code{usbm}. [[DESCRIBE USBM STRUCTURE HERE]].

Random mixing networks can be placed atop the network generated by the process by setting the \code{added.RMCN} to \code{TRUE}. The resulting network will always be a Bernoulli graph with O(N) and Pr(ij)=0.5. 
\end{Details}
%
\begin{Value}
A list containing:
\begin{ldescription}
\item[\code{Results}] A list of seasons each containing an NX5 matrix, the columns of which are the results of the model. 
This length of this list is set by the \code{tfs} parameter in the \code{hyper.params}.
\item[\code{Params}] A character string of the intitial parameters of the model.
\item[\code{Runtime}] How long the model ran (in seconds).
\end{ldescription}
\end{Value}
%
\begin{Note}\relax
Failing to provide a trailing slash ("/" or "\bsl{}", depending on your platform) when specifying \code{output.dir} will result in by your temporary files possibly not being all that temporary :) 
\end{Note}
%
\begin{Author}\relax
Raffaele Vardavas and Christopher Steven Marcum
\end{Author}
\printindex{}
\end{document}
